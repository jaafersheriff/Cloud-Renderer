\newpage
\subsection{Voxel Cone Tracing}\paragraph{}
The next step to render clouds is to ray march through the data. Voxel cone tracing is an existing ray marching technique that utilizes texture mips.

Along with a simple ray march, voxel cone tracing keeps track of the radius of a cone encompassing the ray. 
The cone radius is used to sample different mip levels within the volume. 
The further the ray is sampling, the larger the radius, and therefore the higher the mip level is used. 

% TODO : image 

For our voxel cone tracing method, we render the same billboards as before but this time from the user-camera's perspective. We use the same spherical distribution calculation to find the surface of the sphere for each billboard. This is where we begin our cone trace through the volume. 

We cone trace in the direction of the light source, sampling the volume at higher and higher mip levels as we go. The end result gives us the final shading for our clouds. 
% Code defining voxel cone trace 
\begin{lstlisting}[caption={conetrace\_frag.glsl, 63}]
vec3 startPosition = calculateVoxelPosition(sphereSurfacePosition);
vec3 direction = lightPosition - startPosition;
float shading = coneTrace(startPosition, direction);
...
float coneTrace(vec3 position, vec3 direction) {
    position /= volumeDimension;
    direction /= volumeDimension;

    float color = 0.f;
    for (int i = 1; i <= steps; i++) {
        float coneRadius = coneHeight * tan(coneAngle / 2.f);
        float lod = log2(max(1.f, 2.f * coneRadius));
        vec4 sampleColor = textureLod(volume, position + coneHeight * direction, lod + vctLodOffset);
        color += sampleColor.r * i/steps; // Down scale sample
        coneHeight += coneRadius;
    }
}
\end{lstlisting}\paragraph{}

% cone trace
\begin{figure}[h]
\centering
\includegraphics[width=\textwidth]{../res/conetrace.png}
\caption{Cone tracing results of a single billboard}
\end{figure}

Our cone trace method is fully paramterized. The user can change the number of ray march steps, the angle of cone's head, and the start offset of the cone along the ray. 

